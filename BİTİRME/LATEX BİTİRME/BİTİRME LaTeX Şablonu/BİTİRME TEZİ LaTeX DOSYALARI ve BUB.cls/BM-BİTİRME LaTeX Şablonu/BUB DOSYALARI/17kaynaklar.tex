%kaynaklar ba�l���n� tan�mlar
\renewcommand{\refname}{KAYNAKLAR}

\addcontentsline{toc}{section}{KAYNAKLAR}
\begin{thebibliography}{99}%kaynak ortam� olu�turmak i�in
%%%%%%%%%%%%Kaynak Web sayfas�ndan al�nm�� ise%%%%%%%%%%%%%%%%%
\bibitem{k:1} Wikipedia, \url{https://en.wikipedia.org/wiki/RC4} 

\bibitem{k:2} Ulusal Tez Merkezi, Ar�. G�r. Sefa TUN�ER Kaotik Sistemler Tezi
\bibitem{k:3} \url{http://www.teknokolikler.com/2011/11/c-nedir-c-temelleri-nelerdir.html}
 
\bibitem{k:4} \url{http://www.teknokoliker.com/2011/11/c-nedir-c-temelleri-nelerdir.html}
\bibitem{k:5} \url{https://www.vocal.com/secure-communication/wired-equivalent-privacy-wep/}
\bibitem{k:6} \url{http://www.emo.org.tr/ekler/7600d163fa81512_ek.pdf}
\bibitem{k:7} \url{https://stackoverflow.com/questions/21325661/convert-image-path-to-base64-string}
\bibitem{k:8} Microsoft, \url{https://social.msdn.microsoft.com/Forums/en-US/74fdc1b9-9074-4c49-b90d-fbd1947c2e00/string-to-hexadecimal?forum=Vsexpressvcs} 
\bibitem{k:9} Microsoft, \url{https://msdn.microsoft.com/tr-tr/library/aka44szs(v=vs.110).aspx}

\end{thebibliography}